\documentclass[10pt,a4paper]{article}
\usepackage{amssymb}
\usepackage{lambda,ifthen,calc}
\usepackage{pstricks} %,pstcol,pst-fill}
\usepackage{graphics}
\usepackage{multicol}

\definecolor{a0}{rgb}{.99, .99, .00} %yellow
\definecolor{a1}{rgb}{.77, .77, .00}
\definecolor{a2}{rgb}{.55, .55, .00}
\definecolor{b0}{rgb}{.99, .22, .22} %red
\definecolor{b1}{rgb}{.77, .11, .11}
\definecolor{b2}{rgb}{.55, .00, .00}
\definecolor{c0}{rgb}{.00, .00, .99} %blue
\definecolor{c1}{rgb}{.00, .00, .77}
\definecolor{c2}{rgb}{.00, .00, .55}
\definecolor{d0}{rgb}{.55, .55, .55} %black
\definecolor{d1}{rgb}{.33, .33, .33}
\definecolor{d2}{rgb}{.11, .11, .11}
\definecolor{e0}{rgb}{.99, .99, .99} %white
\definecolor{e1}{rgb}{.77, .77, .77}
\definecolor{e2}{rgb}{.55, .55, .55}


\setlength{\textwidth}{19cm}
\setlength{\marginparwidth}{0cm}
\setlength{\oddsidemargin}{-1in+1cm}
\setlength{\evensidemargin}{-1in+1cm}
\setlength{\textheight}{27.7cm}
\setlength{\topmargin}{0cm}
\setlength{\headheight}{0cm}
\setlength{\headsep}{0cm}
\setlength{\footskip}{0cm}
\setlength{\voffset}{-1in+1cm}
\pagestyle{empty}

\begin{document}

\begin{center}
\begin{tabular}{lcr}
%FFEN_SquareSize 0.5
\scalebox{0.7}{
%FFEN k1K/1#2/1R#1/1#2/1#2
%FFEN_Fill 1
~~%FFEN *3K1*3k/1#2/1*3p#1/1#2/1#2
%FFEN_Fill 0
%FFEN_InnerStyle 1
~~%FFEN A-Bc/D#2/=E+#1/t#2/Sx&
%FFEN_CornersRadius 0.2
%FFEN_InnerStyle 0
~~%FFEN 1#2*3'B/^*1'S1#1*3'O/^*1'T#11*3'X/^*1'O#2*3'E/^*1'D#2*3'S
%FFEN_CornersRadius 0
%FFEN_InnerStyle 0
%FFEN_Fill 2
}
&
\begin{minipage}[b]{6cm}
\begin{center}
{\bf ffen2tex v0.9}\\
{\sl Alain Brobecker, 2012-2015}\\
~\\
\end{center}
\end{minipage}
&
%CUBES_BoxBorder 0
%CUBES_WSize 0.25
%CUBES_DSize 0.25
%CUBES_HSize 0.25
%CUBES 3/1/5/aaa/a../a../a../aaa
~%CUBES 3/1/5/b.b/b.b/b.b/b.b/bbb
~%CUBES 3/1/5/ccc/c.c/ccc/c.c/ccc
~%CUBES 3/1/5/ddd/d../dd./d../ddd
~%CUBES 3/1/5/eee/e../eee/..e/eee
\end{tabular}
\end{center}

\noindent
{\bf What is ffen2tex?}\\
{\bf ffen2tex} is a pre-processor to create {\bf diagrams with square grids} or {\bf cubes setups}
in the \TeX ~typesetting system. For example you can make diagrams for Chess, Checkers, Draughts,
Go, fairy variants of those games and for puzzles...\\
The name comes from FEN (Forsyth-Edwards Notation) which was created to describe Chess positions
(position=diagram+who has the move+castling rights+etc...). In 2002, Joost de Heer designed
the {\bf Fairy FEN}  extension for fairy Chess diagrams (not positions!).
I added some more possibilities for my own purpose.
You will have more informations in the examples or by searching "Forsyth-Edwards Notation" on the internet.\\
~\\


\noindent
{\bf How to use it: pre-processing}\\
The following explanations were written for someone using a window\$/m\$do\$ system.
I hope this example is enough to understand how to use ffen2tex and for people
using another operating system.\\

\noindent
Instead of adding commands to \TeX ~i chose to make a program that must be executed
on your .tex file before you process it with \TeX ~(so it must be pre-processed by
ffen2tex). It converts the \%{FFEN} instructions into pstricks command, so at first
you need to have the standard {\bf pstricks package installed}. You must also
{\bf copy ffen2tex.prg} in your \TeX ~work directory.\\


\noindent
Now create a file named "sample.tex" in your \TeX ~work directory and
write the following in it:\\
\makebox[10cm]{\fbox{
\begin{minipage}[b]{7cm}
\textbackslash documentclass\{article\} \\
\textbackslash usepackage\{pstricks\} \\
\textbackslash begin\{document\} \\
Sample diagram:\textbackslash \textbackslash \\
\%{FFEN} rnbqkp/PKQBNR \\
\textbackslash end\{document\}
\end{minipage}
}}
~\\

\noindent
Then under the m\$do\$ prompt (or the command line of your OS) type the following:\\
\makebox[10cm]{\fbox{
\begin{minipage}[b]{7cm}
c:\textbackslash tex\textgreater ffen2tex sample.tex temporary.tex\\
c:\textbackslash tex\textgreater latex temporary.tex\\
c:\textbackslash tex\textgreater dvips temporary.ps\\
c:\textbackslash tex\textgreater del sample.pdf\\
c:\textbackslash tex\textgreater rename temporary.pdf sample.pdf\\
c:\textbackslash tex\textgreater del temporary.*
\end{minipage}
}}
~\\

\noindent
The resulting "sample.pdf" file contains the following:\\
\makebox[10cm]{\fbox{
\begin{minipage}[b]{7cm}
Sample diagram:\\
%FFEN rnbqkp/PKQBNR
\end{minipage}
}}


~\\
~\\

\noindent
Of course we don't want to type this sequence every time we process
a .tex file, so we will use a batch (shell?) file to do it. I
saved the following batch sequence in "goffen.bat" which is in my
\TeX ~work directory:\\
\makebox[10cm]{\fbox{
\begin{minipage}[b]{7cm}
ffen2tex \%1.tex temporary.tex\\
latex temporary.tex\\
dvips temporary.dvi\\
ps2pdf temporary.ps\\
del \%1.pdf\\
rename temporary.pdf \%1.pdf\\
del temporary.*
\end{minipage}
}}
~\\

\noindent
Then i only have to type this under the m\$do\$ prompt:\\
\makebox[10cm]{\fbox{
\begin{minipage}[b]{7cm}
c:\textbackslash tex\textgreater goffen sample
\end{minipage}
}}
~\\

~\\
~\\

\noindent
Some examples are following. The list of all variables, modifiers and pieces
is in the Quick reference. Happy diagramming...




\newpage
\begin{center}
{\bf ffen2tex v0.9 - \%{FFEN} examples}\\
{\sl Alain Brobecker, august 2015}\\
\end{center}
~\\

\noindent
Since {\bf ffen2tex} is based upon the Forsyth-Edwards Notation,
the default settings are for Chess diagrams. The one below has
been reached after black's third move, how did the game went?\\
\makebox[1cm]{~}
\begin{tabular}{ll}
%FFEN rnb1kbnr/pppp1ppp/8/4p3/3q4/4KP2/PPPPP1PP/RNBQ1BNR
&
\begin{minipage}[b]{12cm}
\%{FFEN} rnb1kbnr/pppp1ppp/8/4p3/3q4/4KP2/PPPPP1PP/RNBQ1BNR
~\\~\\~\\~\\~\\
\end{minipage}
\end{tabular}
~\\
~\\
~\\

\noindent
Diagrams can have any size between $1\* 1$ and $99\* 99$, but for
empty areas of width bigger than 9 squares you'll have to use two
digits or more. Here a width of 10 empty squares is given as two
empty areas of width 5, ie 55, but it could have been 64 or 73...
Below is an international Draughts game with FFEN\_PieceCount set.\\
\makebox[1cm]{~}
\begin{tabular}{ll}
\begin{minipage}{6cm}
%FFEN_PieceCount 1
%FFEN 1D8/2c3c3/55/2c3c1c1/55/8c1/1c3c4/55/5c1c2/55
White plays and wins.
\end{minipage}
&
\begin{minipage}[B]{12cm}
\%{FFEN}\_PieceCount 1\\
\%{FFEN} 1D8/2c3c3/55/2c3c1c1/55/8c1/1c3c4/55/5c1c2/55 \\
~\\~\\~\\~\\
\end{minipage}
\end{tabular}
~\\
~\\

\noindent
We show here a $5\* 5$ part of the first example with only the down border.
The \%{FFEN}\_Fill variable was changed to {\bf 1=odd (ie upper left is filled)} to have
the correct square coloring.
{\bf Note that \%FFEN\_BorderAll is evaluated before the other \%FFEN\_Border modifiers.}\\
\makebox[1cm]{~}
\begin{tabular}{ll}
%FFEN_Fill 1
%FFEN_BorderAll 0
%FFEN_BorderDown 1
%FFEN_PieceCount 0
%FFEN 2p2/1q3/2KP1/PPP1P/BQ1BN
&
\begin{minipage}[b]{12cm}
\%{FFEN}\_PieceCount 0\\
\%{FFEN}\_Fill 1 \\
\%{FFEN}\_BorderAll 0 \\
\%{FFEN}\_BorderDown 1 \\
\%{FFEN} 2p2/1q3/2KP1/PPP1P/BQ1BN \\
\end{minipage}
\end{tabular}
~\\
~\\
~\\


\noindent
To put coordinates we insert letters and figures with the ' escape character
(double ' for text with two characters), and put them outside the board using
the \# modifier. It's tedious but flexible.
{\bf Note that \%{FFEN}\_Fill is still 1, because the
values are not reinitialised between two diagrams.} Upper left corner would be
colored if it were not outside the board.
I added a white circle indicating who has to move, and yes, it's mate in 1.\\
\makebox[1cm]{~}
\begin{tabular}{ll}
%FFEN_BorderAll 0
%FFEN_BorderUp 1
%FFEN_BorderLeft 1
%FFEN #2#'a#'b#'c/#1#'8k2/#1#'73/#C#'6K1R
&
\begin{minipage}[b]{12cm}
\%{FFEN}\_BorderAll 0 \\
\%{FFEN}\_BorderUp 1 \\
\%{FFEN}\_BorderLeft 1 \\
\%{FFEN} \#2\#'a\#'b\#'c/\#1\#'8k2/\#1\#'73/\#C\#'6K1R
\end{minipage}
\end{tabular}
~\\
~\\
~\\

\noindent
With the *N modifier we can rotate the pieces,
this is often used for fairy pieces in Chess problems.\\
\makebox[1cm]{~}
\begin{tabular}{ll}
%FFEN_Fill 2
%FFEN_BorderAll 1
%FFEN k*1k*2k*3k/'a*1'b*2'c*3'd/v*1v*2v*3v/K*1K*2K*3K
&
\begin{minipage}[b]{12cm}
\%{FFEN}\_Fill 2 \\
\%{FFEN}\_BorderAll 1 \\
\%{FFEN} k*1k*2k*3k/'a*1'b*2'c*3'd/v*1v*2v*3v/K*1K*2K*3K \\
\end{minipage}
\end{tabular}



\newpage
\noindent
By modifying the size, the inner lines, the borders and the filling, we leave the
world of Chess/Draughts/Checkers for Tic-tac-toe. What was the last move here?\\
\makebox[1cm]{~}
\begin{tabular}{ll}
%FFEN_SquareSize 0.7
%FFEN_Fill 0
%FFEN_InnerStyle 1
%FFEN_InnerWidth 0.08
%FFEN_BorderStyle 0
%FFEN ..x/.x./xxx
&
\begin{minipage}[b]{12cm}
\%{FFEN}\_SquareSize 0.7\\
\%{FFEN}\_Fill 0 \\
\%{FFEN}\_InnerStyle 1 \\
\%{FFEN}\_InnerWidth 0.08 \\
\%{FFEN}\_BorderStyle 0 \\
\%{FFEN} ..x/.x./xxx
\end{minipage}
\end{tabular}
~\\
~\\

\noindent
White pieces can be colored with the - or = modifiers.
Pieces can be circled with the @ modifier.
And using the \# modifier we can make odd-shaped boards
(it can also apply on many empty squares).\\
\makebox[1cm]{~}
\begin{tabular}{ll}
%FFEN_SquareSize 0.5
%FFEN_InnerStyle 3      %FFEN_InnerWidth 0.03
%FFEN_BorderStyle 1
%FFEN #1@'1#2/@'32-N/#2@'2#1
&
\begin{minipage}[b]{12cm}
\%{FFEN}\_SquareSize 0.5\\
\%{FFEN}\_InnerStyle 3 ~~~~~~ \%{FFEN}\_InnerWidth 0.03 \\
\%{FFEN}\_BorderStyle 1\\
\%{FFEN} \#1@'1\#2/@'32-N/\#2@'2\#1\\
\end{minipage}
\end{tabular}
~\\
~\\

\noindent
With the $\wedge$ and \textgreater modifiers it's possible to put pieces inbetween
squares. This is used in Chess for some retrograde analysis problems, but the most
obvious use is for Go
(even better: use the {\bf igo} package by \'Etienne Dupuis). Can white kill here?\\
\makebox[1cm]{~}
\begin{tabular}{ll}
%FFEN_BorderRight 0
%FFEN_BorderUp 0
%FFEN_InnerStyle 1
%FFEN_InnerWidth 0.015
%FFEN 7/1^>C^>C^>C3/^>C1^>c^>c^>C^>C1/^>c1^>c1^>c^>C1/#4#^>c#2
&
\begin{minipage}[b]{12cm}
\%{FFEN}\_BorderRight 0 \\
\%{FFEN}\_BorderUp 0 \\
\%{FFEN}\_InnerStyle 1 \\
\%{FFEN}\_InnerWidth 0.015 \\
\%{FFEN} 7/1$\wedge$\textgreater C$\wedge$\textgreater C$\wedge$\textgreater C3/$\wedge$\textgreater C1$\wedge$\textgreater c$\wedge$\textgreater c$\wedge$\textgreater C$\wedge$\textgreater C1/$\wedge$\textgreater c1$\wedge$\textgreater c1$\wedge$\textgreater c$\wedge$\textgreater C1/\#4\#$\wedge$\textgreater c\#2 \\
\end{minipage}
\end{tabular}
~\\


\noindent
The \$ modifier changes the fill property of a particular square
(it can also apply on many empty squares and to outside squares).
Here you must fill the empty squares with numbers 1 to 8.\\
\makebox[1cm]{~}
\begin{tabular}{ll}
%FFEN_FillStyle 2
%FFEN_InnerStyle 1
%FFEN_InnerWidth 0.03
%FFEN_BorderAll 1
%FFEN 1'+1'=1/'+$1'-$1'+/1'-1'=1/'=$1'=$1'=/1'+1'=''10
&
\begin{minipage}[b]{12cm}
\%{FFEN}\_FillStyle 2 \\
\%{FFEN}\_InnerStyle 1 \\
\%{FFEN}\_InnerWidth 0.03 \\
\%{FFEN}\_BorderAll 1 \\
\%{FFEN} 1'+1'=1/'+\$1'-\$1'+/1'-1'=1/'=\$1'=\$1'=/1'+1'=''10\\
\end{minipage}
\end{tabular}
~\\
~\\

\noindent
With the \%{FFEN}\_CornersRadius and the [, ], \~ ~ and \_ modifiers you can make
"Dots and boxes" diagrams. What to play here to win?\\
\makebox[1cm]{~}
\begin{tabular}{ll}
%FFEN_SquareSize 1
%FFEN_InnerStyle 0
%FFEN_BorderWidth 0.1
%FFEN_CornersRadius 0.15
\scalebox{0.5}{
%FFEN 2[12/1[1_1[_1_1/1[_11[11/1[11[1[1/3[11
}
&
\begin{minipage}[b]{12cm}
\%{FFEN}\_SquareSize 1 \\
\%{FFEN}\_InnerStyle 0\\
\%{FFEN}\_BorderWidth 0.1  \\
\%{FFEN}\_CornersRadius 0.15  \\
\textbackslash scalebox\{0.5\}\{ \%{FFEN} 2[12/1[1\_1[\_1\_1/1[\_11[11/1[11[1[1/3[11\\
\}\\
\end{minipage}
\end{tabular}
~\\
~\\

\noindent
Finally you can use the \{ and \} delimiters and directly insert psscript code
to create whatever symbol which is not yet present. No space is allowed between \{ and \},
but you can create \TeX ~macros to put inside as in the following example.\\
\makebox[1cm]{~}
\begin{tabular}{ll}
\def\strip{\psline[linewidth=0.03](0,0)(0.3,0.3)(0.7,0.3)(1,0)}
%FFEN_CornersRadius 0
%FFEN_BorderWidth 0.03
%FFEN_InnerStyle 1
\scalebox{0.9}{
%FFEN #1#{\strip}1#{\strip}#{\strip}/#*1{\strip}4/#1#*2{\strip}1#*2{\strip}#*2{\strip}
}
&
\begin{minipage}[b]{12cm}
\textbackslash def\textbackslash S\{\textbackslash psline[linewidth=0.03](0,0)(0.3,0.3)(0.7,0.3)(1,0)\}\\
\%{FFEN}\_CornersRadius 0 \\
\%{FFEN}\_CornersRadius 0\\
\%{FFEN}\_BorderWidth 0.03\\
\%{FFEN}\_InnerStyle 1\\
\%{FFEN} \#1\#\{\textbackslash S\}1\#\{\textbackslash S\}\#\{\textbackslash S\}/\#*1\{\textbackslash S\}4/\#1\#*2\{\textbackslash S\}1\#*2\{\textbackslash S\}\#*2\{\textbackslash S\}\\
\end{minipage}
\end{tabular}
~\\
~\\

\noindent
{\bf Some more remarks:}\\
\textbullet~~ FFEN\_ThinLineWidth, FFEN\_BoldLineWidth, FFEN\_InnerWidth and
FFEN\_BorderWidth and everything related to them are relative to FFEN\_SquareSize,
so by default the real width for FFEN\_ThinLineWidth is 0.03$\*$0.5=0.015 cm.
\\
\textbullet~~ The width of Bars and Diagonals is the same as FFEN\_BorderWidth.
\\
\textbullet~~ The centered dots (FFEN\_FillStyle 6) have diameter of 3 times FFEN\_BorderWidth.
\\
\textbullet~~ FFEN\_BoldLineWidth is only used for Bold Cross (x) and Bold Circle (.).
\\
\textbullet~~ When FFEN\_Fill is set to 2 (even), the upper left corner is empty,
when it's set to 1 (odd) the upper left corner is filled.
\\
\textbullet~~ Symbols, figures and letters are inserted using
"\textbackslash rput[B](0.5*FFEN\_SquareSize,0.25*FFEN\_SquareSize)\{42\}",
so they look best with a FFEN\_SquareSize of 0.5, size being
changed afterward with a \textbackslash scalebox.


\newpage
\begin{center}
{\bf ffen2tex v0.9 - Quick  \%{FFEN} reference}\\
{\sl Alain Brobecker, august 2015}
\end{center}

%\noindent
%{\bf Usage:} ffen2tex~~in.tex~~out.tex\\

\noindent
{\bf Draw a diagram:} \%{FFEN UpperRow/.../LowerRow}\\
%\begin{tabular}{l}
%\%{FFEN rowN/.../row2/row1}\\
%\end{tabular}

\noindent
{\bf Board Modifiers:} {\sl (default values are given, not reinitialised between two diagrams)}\\
\begin{tabular}{ll}
\%FFEN\_SquareSize 0.5 & \\
\%FFEN\_ThinLineWidth 0.03 & \\
\%FFEN\_BoldLineWidth 0.15 & \\
\%FFEN\_Fill 2  & {\sl (0=none / 1=odd / 2=even (Chess) / 3=full)} \\
\%FFEN\_FillStyle 0  & {\sl (0=lightgray / 1=gray / 2=black / 3=vlines / 4=hlines / 5=crosshatch / 6=centered dot)} \\
\%FFEN\_InnerStyle 0  & {\sl (0=none / 1=solid / 2=dashed / 3=dotted)} \\
\%FFEN\_InnerWidth 0.03 & \\
\%FFEN\_BorderStyle 1 & {\sl (0=none / 1=solid / 2=dashed / 3=dotted)} \\
\%FFEN\_BorderWidth 0.06 & \\
\%FFEN\_BorderUp 1 &{\sl (0=off / 1=on)} \\
\%FFEN\_BorderLeft 1 & {\sl (0=off / 1=on)} \\
\%FFEN\_BorderDown 1 & {\sl (0=off / 1=on)} \\
\%FFEN\_BorderRight 1 & {\sl (0=off / 1=on)} \\
\%FFEN\_BorderAll 1 & {\sl (0=off / 1=on, affects all 4 borders variables)} \\
\%FFEN\_PieceCount 0 &  {\sl (0=off / 1=W+B / 2=?+?=W+B)} \\
\%FFEN\_CornersRadius 0 &  {\sl (0=off / ?=radius)} \\
%\%FFEN\_AllOptions 0 &  {\sl (0=off / ?=radius)} \\
\end{tabular}
I also added the \%FFEN\_HBorderStyle, \%FFEN\_VBorderStyle, \%FFEN\_HBorderWidth, \%FFEN\_VBorderWidth,\\
\%FFEN\_HInnerStyle, \%FFEN\_VInnerStyle, \%FFEN\_HInnerWidth and \%FFEN\_VInnerWidth modifiers.\\
{\bf Warnings:} If you change more than once a value before using \%{FFEN} only the first one
is taken in account. The values are not reinitialised after a diagram.
\%FFEN\_BorderAll is evaluated before the other \%FFEN\_Border modifiers.\\


\noindent
{\bf Pieces:}\\
%FFEN_SquareSize 0.5
%FFEN_ThinLineWidth 0.03
%FFEN_BoldLineWidth 0.15
%FFEN_Fill 2
%FFEN_FillStyle 0
%FFEN_InnerStyle 0
%FFEN_InnerWidth 0.03
%FFEN_BorderStyle 1
%FFEN_BorderWidth 0.06
%FFEN_BorderAll 1
\begin{tabular}{c|c|ccc|c|ccc|c|c}
char & piece & output && char & piece & output && char & piece &output
~\\
K / k & W / B King & %FFEN K
~ %FFEN k
&& \& & Heart  & %FFEN &
&& u & Up Bar & %FFEN u
~\\
Q / q & W/B Queen & %FFEN Q
~ %FFEN q
&& " & Diamond  & %FFEN "
&& , & Small Vertical Bar & %FFEN ,
~\\
R / r & W / B Rook & %FFEN R
~ %FFEN r
&& ! & Spade  & %FFEN !
&& \textbar & Vertical Bar  & %FFEN |
~\\
B / b & W / B Bishop & %FFEN B
~ %FFEN b
&& ? & Club  & %FFEN ?
&& v & Up Left Bar  & %FFEN v
~\\
N / n & W / B kNight & %FFEN N
~ %FFEN n
&& A & W Up Left Arrow & %FFEN A
&& y & Vertical Left Bar  & %FFEN y
~\\
P / p & W / B Pawn & %FFEN P
~ %FFEN p
&& a & W Up Arrow & %FFEN a
&& + & Cross Bars & %FFEN +
~\\
C /c & W / B Circle & %FFEN C
~ %FFEN c
&& . & Bold Circle & %FFEN .
&& U & Half Diagonal  & %FFEN U
~\\
D / d & W / B Dame & %FFEN D
~ %FFEN d
&& & &
&& \textbackslash & Diagonal & %FFEN \
~\\
O / o & W / B Small Circle  & %FFEN O
~ %FFEN o
&& & &
&& V & V Half Diagonals & %FFEN V
~\\
T / t & W / B Triangle & %FFEN T
~ %FFEN t
&& &  &
&& Y & Y Diagonal & %FFEN Y
~\\
S / s & W / B Square & %FFEN S
~ %FFEN s
&& & &
&& \% & Cross Diagonals  & %FFEN %
~\\
X / x & W / B Cross & %FFEN X
~ %FFEN x
&& : & Thin Cross & %FFEN :
&& $<$ & Less Than & %FFEN <
~\\
E / e & W / B Star  & %FFEN E
~ %FFEN e
&& 'a & Symbol & %FFEN 'a
&& ''42 & Double Symbol & %FFEN ''42
~\\
L / l & V / H Line Fill & %FFEN L
~ %FFEN l
&& F & CrossHatch Fill & %FFEN F
&& f & White Fill  & %FFEN f
\end{tabular}
~\\
Valid symbols are !"\#\$\%\&()*+,-./0..9:;\textless =\textgreater ?@A..Z[\textbackslash]$\wedge$\_`a..z
~\\
{\bf Double symbols are inserted using two ' characters, not the double quote!}\\
{\bf TeX instructions and macros can be inserted as a piece using the \{ and \} delimiters (no space!).}\\


\noindent
{\bf Piece Modifiers:}\\
\begin{tabular}{cl}
\textgreater & piece is between this square and right square\\
$\wedge$ & piece is between this square and up square\\
$\ast$N  & rotated N*90 degree anticlockwise with N $\in$ [0;3]\\
- or =& lightgray fill or gray fill (only for white pieces, $\heartsuit$ and $\diamondsuit$ are not white)\\
@ & piece is circled (same size as White Circle, but transparent)
\end{tabular}
~\\

\noindent
{\bf Square Modifiers:} {\sl (can apply on many empty squares)}\\
\begin{tabular}{cl}
\$ & change square(s) fill property\\
\# & square(s) outside board (coloring affected by \$ but not by \%FFEN\_Fill)\\
$[$ and/or $]$ & border on the left and/or right of next square(s)\\
\~ ~and/or \_ & border above and/or below next square(s)\\
%? & (1 bit) & square is circled\\
\end{tabular}

\newpage
\begin{center}
{\bf ffen2tex v0.9 - Quick \%{CUBES} reference and examples}\\
{\sl Alain Brobecker, august 2015}
\end{center}

\noindent
{\bf Cubes Modifiers:} {\sl (default values are given, not reinitialised between two diagrams)}\\
\begin{tabular}{ll}
\%CUBES\_WAngle 30  &(between [0;90])\\
\%CUBES\_DAngle 30  &(between [0;90])\\
\%CUBES\_WSize 1.0  &\\
\%CUBES\_DSize 1.0  &\\
\%CUBES\_HSize 1.0  &\\
\%CUBES\_Rotation 0 &(in [0;3] rotation around the vertical axis)\\
\%CUBES\_UpsideDown 0 &(1=mirror around the vertical axis, made after rotation)\\
\%CUBES\_BoxBorder 000000000011 & (bit 0,1=down, 2,3=backleft, 4,5=backright, 6,7=up, 8,9=frontright, 10,11=frontleft)\\
\%CUBES\_BoxInner 000000000011 & (idem, those are binary numbers, bit 0 on the right, no need to put the leftmost zeroes)\\
\%CUBES\_BoxBorderStyle 1 &(0=none / 1=solid / 2=dashed / 3=dotted)\\
\%CUBES\_BoxInnerStyle 1 &(0=none / 1=solid / 2=dashed / 3=dotted)\\
\%CUBES\_BoxBorderWidth 0.06 &\\
\%CUBES\_BoxInnerWidth 0.03 &\\
\%CUBES\_CubesWidth 0.03 &
\end{tabular}

~\\
\noindent
{\bf Draw cubes:}\\
\%{CUBES} W/D/H/.../.../.../.../.../.../.../.../...
~\\
\begin{tabular}{ll}
. & empty\\
0-9 & plain yellow/red/blue/black/white/green/orange/gray/cyan/magenta cube\\
%# for hatched cube
a-z & for cubes with triple of colors defined beforehand in the file with:\\
 & \makebox[1cm]{~}\textbackslash definecolor\{a0\}\{rgb\}\{.99, .99, .99\} \%left\\
 & \makebox[1cm]{~}\textbackslash definecolor\{a1\}\{rgb\}\{.33, .33, .33\} \%right\\
 & \makebox[1cm]{~}\textbackslash definecolor\{a2\}\{rgb\}\{.66, .66, .66\} \%up\\
\end{tabular}


~\\
\noindent
{\bf Examples:}\\
Here is a sofa with Width=5, Depth=3 and Height=3. It uses a user-defined color.\\
\begin{tabular}{ll}
\definecolor{a0}{rgb}{.4, .1, .5}
\definecolor{a1}{rgb}{.6, .3, .7}
\definecolor{a2}{rgb}{.8, .5, .9}
%CUBES_BoxBorder 0
%CUBES_WSize 0.5
%CUBES_DSize 0.5
%CUBES_HSize 0.5
%CUBES 5/3/3/aaaaa/...../...../aaaaa/a...a/...../aaaaa/aaaaa/aaaaa
&
\begin{minipage}[b]{12cm}
\textbackslash definecolor\{a0\}\{rgb\}\{.4, .1, .5\}\\
\textbackslash definecolor\{a1\}\{rgb\}\{.6, .3, .7\}\\
\textbackslash definecolor\{a2\}\{rgb\}\{.8, .5, .9\}\\
\%CUBES\_BoxBorder 0\\
\%CUBES\_WSize 0.5\\
\%CUBES\_DSize 0.5\\
\%CUBES\_HSize 0.5\\
\%{CUBES} 5/3/3/aaaaa/...../...../aaaaa/a...a/...../aaaaa/aaaaa/aaaaa\\
\end{minipage}
\end{tabular}

~\\
\noindent
Almost the same sofa from another angle, and with predefined colors.\\
\begin{tabular}{ll}
%CUBES_WAngle 0
%CUBES_DAngle 45
%CUBES_WSize 0.5
%CUBES_DSize 0.35355339
%CUBES_HSize 0.5
%CUBES 5/3/3/01234/...../...../12345/...../...../45678/23456/34567
&
\begin{minipage}[b]{12cm}
\%CUBES\_WAngle 0\\
\%CUBES\_DAngle 45\\
\%CUBES\_WSize 0.5\\
\%CUBES\_DSize 0.35355339  \% divided by sqrt(2)\\
\%CUBES\_HSize 0.5\\
\%{CUBES} 5/3/3/01234/...../...../12345/...../...../45678/23456/34567\\
\end{minipage}
\end{tabular}

~\\
\noindent
The BoxBorder and BoxInner modifiers allow to have grids or lines around the figure:\\
\begin{tabular}{ll}
%CUBES_WAngle 10
%CUBES_DAngle 40
%CUBES_WSize 0.6
%CUBES_DSize 0.48
%CUBES_HSize 0.6
%CUBES_BoxBorder 111111
%CUBES_BoxInner 101011
%CUBES 5/3/3/...../..1../...../...../.01../...../...../.012./.....
&
\begin{minipage}[b]{12cm}
\%CUBES\_WAngle 10\\
\%CUBES\_DAngle 40\\
\%CUBES\_WSize 0.6\\
\%CUBES\_DSize 0.48\\
\%CUBES\_HSize 0.6\\
\%CUBES\_BoxBorder 111111\\
\%CUBES\_BoxInner 101011\\
\%{CUBES} 5/3/3/...../..1../...../...../.01../...../...../.012./.....\\
\end{minipage}
\end{tabular}


\end{document}